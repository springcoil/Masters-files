\section{Berry Phase}
\subsection{Introduction}
The following is based on an amalgamation of notes made from \cite{2005math.ph_geometricphases} and 
the canonical textbook for learning about the Mathematics of the Geometric phase is ,\cite{geometricphases2004}.

We invite the reader to perform the following simple experiment. Put your
arm out in front of you keeping your thumb pointing up perpendicular to your
arm. Move your arm up over your head, then bring it down to your side, and
at last bring the arm back in front of you again. In this experiment an object
(your thumb) was taken along a closed path traced by another object (your
arm) in a way that a simple local law of transport was applied. In our case
the local law consisted of two ingredients: (1) preserve the orthogonality of
your thumb with respect to your arm and (2) do not rotate the thumb about
its instantaneous axis (i.e your arm). Performing the experiment, in this way
you will manage to avoid rotations of your thumb locally, however in the end
you will experience a rotation of 90\textdegree globally.
The experiment above can be regarded as the archetypical example of the
phenomenon called anholonomy by physicists and holonomy by mathematicians.
In this paper we consider the manifestation of this phenomenon in the
realm of quantum theory. The objects to be transported along closed paths
in suitable manifolds will be wave functions representing quantum systems.
After applying local laws dictated by inputs coming from physics, one ends up
with a new wave function that has picked up a complex phase factor. Phases
of this kind are called Geometric Phases with the famous Berry Phase being
a special case.

\subsection{Space of Rays}
Let us consider a quantum system with physical states represented by elements $|\phi\rangle$
of some Hilbert space $\H$ with scalar product $\langle | \rangle: \H \times \H \to \Cbb$.
For simplicity we assume that $\H$ is finite dimensional $\H \iso \Cbb^{n+1}$ with $n \gt 1$.
The infinite dimensional case can be studied by taking the inductive limit $n \to \infty$. Let us denote the complex
amplitudes characterizing the state $| \psi \rangle$ by $Z^{\alpha}$, $\alpha = 0,1,\cdots, n$
For a normalized state we have 
\begin{equation}\label{a}
 \|\psi\|^{2} = \langle \psi | \psi \rangle \equiv \delta_{\alpha \beta} \bar{Z}^{\alpha} Z^{\beta} \equiv
\bar{Z}_{\alpha} Z^{\alpha} = 1,
\end{equation}
where summation over repreated indices is understood, indices are raised and lowered by $\delta^{\alpha \beta}$ and
$\delta_{\alpha \beta}$ respectively, and the overbar refers to complex conjugation. 
 A normalised state lies on the unit sphere $\S \iso S^{2n+1}$ in $\Cbb^{n+1}$. Two nonzero states $|\psi\rangle$ 
and $|\phi\rangle$ are equivalent $| \psi \rangle \sim |\phi \rangle$ iff they are related as \psiket = $\lambda \phiket$
for some nonzero complex number $\lambda$. For equivalent states quanities like 
\begin{equation}\label{b}
 \dfrac{\psibra \mathbf{A} \psiket}{\psibra \psi \rangle}, \dfrac{|\psibra \phiket|^{2}}{\|\psi\|^{2}\|\phi\|^{2}}
\end{equation}
having physical meaning (mean value of a physical quantity represented by a Hermitian operator $\mathbf{A}$, transition
probability from a physical state represented by \psiket to one represented by \phiket) are invariant. 
Hence the real space of states representig the physical states of a quantum system unambigiously is the set of 
equivalence classes \P\; \;$\equiv \H / \sim$ $\P\;$ is called the \textit{space of rays}.
For $\H \iso \Cbb^{n+1}$ we have $\P\; \iso \Cbb \P\;bb^{n}$, where $\Cbb \P\;bb^{n}$ is the n-dimensional complex projective
space. For normalised states $\psiket$ and $\phiket$ are equivalent iff $\psiket = \lambda \phiket$, 
where $|\lambda|=1$ i.e. $\lambda \in U(1)$. In other words, two normalised states are equivalent iff they differ
merely in a complex phase. 
  \paragraph{} 
It is well known that \S \;can be regarded as the total space of a principal bundle \P\; with the structure group U(1). This meanas
that we have the projection 
\begin{equation}\label{c}
 \pi: \psiket \in \S \subset \H \to \psiket \psibra \in \P\;,
\end{equation}
where the rank one projector $$\psiket \psibra$$ represents the equivalence class of $\psiket$. Since we will use this 
bundle frequencly we will call it $\eta_1$.
Then we have 
\begin{equation}\label{d}
 \eta_1 : U(1) \hookrightarrow \S \stackrel{\pi}{\to} \P\;
\end{equation}
For $Z^{0} \neq 0$ our space of rays \P\; can be given in local coordinates as 
\begin{equation}\label{e}
 \omega^{j} \equiv \dfrac{Z^{j}}{Z^{0}}, \;j=1,\cdots,n
\end{equation}
The $\omega^{j}$ are inhomogeneous co-ordinates for $\Cbb\Pbb^{n}$ on the coordinate patch $\mathcal{U}_{0}$
defined by the condition $Z^{0} \neq 0$.
\paragraph{}
 \P\; is a compact complex manifold with a natural Riemannian metric g. This metric g is induced from the scalar 
product on \H. Let us motivate the constructio of g by using the physical input provided by the invariance of the transition
probability of \ref{a}. For this we define a \textbf{distance} between $\psiket \psibra$ and $\phiket \phibra$ in 
\P\; as follows 
\begin{equation}\label{f}
 \cos^{2}(\delta (\psi, \phi)/2) \equiv \dfrac{|\psibra \phiket|^2}{\|\phi\|^{2} \|\psi\|^2}
\end{equation}
This defnition makes sense due to the Cauchy-Schwarz inequality applied to the R.H.S of \ref{f} is non-negative and less than
or equal to one. It is equal to one iff $$\psiket$$ is a nonzero complex multiple of $\phiket$ i.e. iff they define
the same point in \P\;. Hence in this case $\delta(\psi, \phi) = 0$ as we expected. 
 \paragraph{}
Suppose now that $\psiket$ and $\phiket$ are seperated by an infinitesimal distance $ds = \delta(\psi, \phi)$.
Putting this into the definition \ref{f}, using the local coordinates $\omega^j$ for $\psiket$ and $\omega^{j} + d\omega^{j}$
for $\phiket$ after Taylor expanding both sides one gets\footnote{The author was unable to replicate this step} 
\begin{equation}\label{g}
 ds^2 = 4g_{j\bar{k}}d\omega^j d\omega^{\bar{k}}, j,\bar{k} = 1,2,\cdots, n
\end{equation}
  where 
\begin{equation}\label{h}
 g_{j\bar{k}} \equiv \dfrac{(1+ \bar{\omega_{l}}\omega^l)\delta_{jk} - \bar{\omega}_{j} \omega_{k}}{(1 + \bar{\omega}
_{m} \omega^{m})^{2}}
\end{equation}
with 
$$d \omega^{\bar{k}} \equiv d \bar{\omega}^k$$. The line element \ref{g}) defines the \textbf{Fubini-Study metric} for 
$\P\;$.
\subsection{The Pancharatnam Connection}
Having defined our basic entity the space of rays $\P\;$ and the principal U(1) bundle $\eta_1$ now we define a 
connection giving rise to a local law of parallel transport. In the mathematical literature the connection we are 
going to define is called the canonical connection on our principal bundle.
\paragraph{}
 The information we need is an adaptation of Pancharatnam's study of polarized light to quantum mechanics. 
Let us consder two normalized states $\psiket$ and $\phiket$. When these states belong to the same ray then we have
$\psiket = e^{i\theta}\phiket$ for some phase fact $e^{i \theta}$, hence the phase difference is $\theta$
How do we work out the phase difference between $\psiket$ and 
$\phiket$ (not orthogonal) when the states belong to \textit{different rays?} To compare the phases of non-orthogonal
states belonging to different rays Pancharatnam employed the following simple rule: two states are 'in phase' iff their
inference is maximal. In order to find the state $\phiket \equiv e^{i\theta}|\phi^{'} \rangle $ from the ray spanned
by the representative $|\phi^{'} \rangle$ which is in phase with \psiket we have to find a $\theta$ modulo $2\pi$
for which the interference term in 
\begin{equation}\label{i}
 \|\psi + e^{i\theta} \phi^{'}\|^{2} = 2 (1 + \mathit{Re}(e^{i\theta}\psibra |\phi^{'} \rangle))
\end{equation}
is maximal. Obviously the interference is maximal iff 
$(e^{i\theta}\psibra |\phi^{'} \rangle))$ is a real positive number i.e. 
\begin{equation}\label{j}
 e^{i \theta} = \dfrac{\langle \phi^{'} \psiket}{|\langle \phi^{'} \psiket|}, \phiket = |\phi^{'} \dfrac{\rangle \phi^{'} | \psi \rangle}
{|\langle \phi^{'}| \psi \rangle |} 
\end{equation}
Hence for the state $\phiket$ 'in phase' with $\psiket$ we have 
\begin{equation}\label{k}
 \psibra \phi \rangle = |\psibra \phi^{'} \rangle| \in \Rbb^{+}
\end{equation}
