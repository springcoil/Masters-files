\section{Connections and the Cartan formalism}
\begin{quotation}
 On a manifold it is necessary to use covariant differentation, curvature measures its noncommutativity. Its combination 
as a characteristic form measures the nontriviality of the underlying bundle. This train of ideas is so simple, that their
importance can not be exaggerated. - Shiing-shen Chern
\end{quotation}

For a smooth (for now, real) manifold M, we let $\AA(M)$ denote smooth \Cbb-valued differential forms and \AAA(M) denote
smooth functions. Similarly for a complex bundle $E \to M$, we let E(M) denote (smooth, \Cbb-valued ) sections
\subsection{Connections}
Recall that a connection on a vector bundle E over a smooth manifold is a \Cbb-homomorphism 
$$ \nab:E(M) \rightarrow (\AA \tensor E)(M)$$ 
that maps global sections on M to global sections of $\AA \tensor E$, which satisfies the Leibniz rule
$ \nab(fs) = (df)s + f \nab s$   $f \in \AAA(M)$  $s \in E(M)$
This is essentially an way of differentiating sections of E, because for any vector field X on M, we can define 
\begin{definition}
The Covariant derivative w.r.t this connection of s in the direction of X
 $$\nab_X s$$. This satisfies
\begin{enumerate}
 \item $\nab_fX(s) \;=\; f\nab_X s$
\item $\nab_X(fs)\;=\;(Xf)_s\;+\;f\nab_X s$
\end{enumerate}

\end{definition}
In fact these two properties {\bf characterize} a connection.
\\ We can describe a connection {\bf locally} in terms of frames.
\begin{definition}
 Recall that a {\bf frame} of an n-dimensional vector bundle E, over an open subset $U \subset M$, is a family of sections 
$(e_1,\cdots e_n)\;\in\;E(U)$ that form a basis at each point; thus ${e_1,\cdots,e_n}$ forms a vector bundle isomorphism
 between
$E|_U$ and the trivial bundle. 
\end{definition}
Then $\nab$ is {\bf determined} over U by the elements $\nab_{e_1},\cdots \nab_{e_n}\;\in\;(\AA \tensor E)(U)$. For any 
sections 
s of $E(U)$ can be written as $s \; = \;\sum_i f_i e_i$ for the $f_i$ smooth functions, and consequently 
$$\nab s = \sum e_i(df_i)\;+\;\sum f_i \nab e_i$$
In other words, if we use the fram ${e_i}$ to identify each section of E(U) with the tuple 
$f_i$ such that $s\;=\;\sum f_i e_i$ then $\nab$ acts by applying d and multiplying by suitable matrix corresponding to the $\nab e_i$.
In view of this we make:
\begin{definition}
Given a frame $\Fff = {e_1,\cdots,e_n}$ over U and a connection $\nab$, we define the n-by-n matrix $\theta(\Fff)$ of 
1-forms via
$$\nab \Fff = \theta(\Fff)\Fff$$
In other words, $\nab e_i\;=\;\sum_{j}\theta(\Fff)_{ij} e_j$ for each j
\end{definition}
Note that the $\theta$ itself makes no reference to the bundle: it is simply a matrix of 1-forms.
Given a frame $\Fff$, and given $g:U \rightarrow GL_n(\Cbb)$, we define a new frame $g\Fff$ by multiplying on the left.
We would like to determine how a connection {\bf transforms} with respect to a change of frame, so we can think of a 
connection
in a different way. Namely we have:
$$\nab (g \Fff)\;=\;(dg)\Fff \;+\;g\nab \Fff\;=\;(dg)\Fff\;+\;g\theta(\Fff)\Fff$$
where df is considered as a matrix of 1-forms. As a result we get the {\bf transformation law}
\begin{equation} \label{transformation law}
 \theta(g\Fff) = (dg)g^{-1} + g\theta(\Fff)g^{-1},\;g:U\to\;GL_n(\Cbb)
\end{equation}
Conversely, if we have for each local fram $\Fff$ of a vector bundle $E\to M$ a matrix $\theta(\Fff)$ of 1-forms as above,
which satisfy the transformation law \ref{transformation law} as above,then we get a connection on E.
\begin{proposition}
 Any vector bundle $E\to M$ admits a connection
\end{proposition}
\begin{proof}
 It is easy to see that a convex combination of connections is a connection. Namely in each coordinate patch U over which E 
is
 trivial
with a fixed frame, we choose the matrix $\theta$ arbitrarily and get some connection $\nab'_U$ on $E|_U$. Let these 
various 
$U'_s$ form an open cover $\mathfrak{A}$.
   Then we can find a partition of unity $\phi_U$, $U\in\mathfrak{A}$ subordinate to $\mathfrak{A}$, and we can define our
 global
connection va 
$$\nab \;=\;\sum_U \phi_U \nab'_U $$
\end{proof}

\subsection{Curvature}
We want to now describe the {\bf curvature} of a connection. A connection is a means of differentiating sections; 
however, it 
may not satisfy the standard results for functions, that mixed partials are equal. The curvature will be the measure of how
 much
that fails. Let M be a smooth manifold, $E\to M$ a smooth complex vector bundle. Given a connection $\nab$ on E 
,the curvature is going to be a global section of $\AAA^{2}\tensor\hom(E,E)$: in other words, the global differential
2-forms with coefficients in the vector bundle $\hom(E,E)$

\begin{proposition}
 Let s be a section of E, and X,Y vector fields.
The map:
$$s,X,Y \mapsto R(X,Y,s)=(\nab_Y \nab_X - \nab_X \nab_Y - \nab_{[X,Y]})s$$
is a bundle map $E \to E$, and is $\AAA(M)$-linear in X,Y
\end{proposition}

\begin{proof}
 Calculation typically done to define the Riemannian curvature tensor in the case of the tangent bundle
\end{proof}
Since the quantity $R(X,Y,s)$ is $\AAA(M)$-linear in all these quantities (X,Y,s),and clearly alternating in X,Y, we 
can think of it as a global section of the bundle $\AAA^{2}\tensor \hom(E,E)$. Here recall that $\AAA^{2}$ is the
bundle of 2-forms.
\begin{definition}
 The above elements of $\AAA^{2}\tensor \hom(E,E)(M)$ is called the \textbf{curvature} of the connection $\nab$ and is 
denoted by $\Theta$.
\end{definition}

We now wish to think of the curvature in another manner. To do this, we start by extending the connection $\nab$ to maps 
$\nab:(E \tensor \AAA^p)(M)\to (E\tensor \AAA^{p+1})(M)$. The requirement is that the Leibnitz rule holds: that is,
\begin{equation}\label{Leibnitz}
\nab (\omega s) = (d \omega)s + (-1)^p \omega \wedge \nab s,
\end{equation}
whenever $\omega$ is a p-form and s a global section. We can this locally and glue them.
Thus: 
\begin{proposition}
 One can extend $\nab$ to map s 
$$\nab: (E \tensor \AAA^p)(M)\to (E\tensor \AAA^{p+1})(M) $$ satisfying \ref{Leibnitz}
\end{proposition}
Given such an extension, we can consider the map
$$\nab^2:E(M) \to (E \tensor \AAA^2(M)) $$.
This is $\AAA(M)$-linear. Indeed we can check this by computation
\begin{example}
 $$\nab^2(fs) = \nab(\nab(fs))
\\ = \nab(dfs + f\nab s)
\\ = d^2 fs +(-1)df \nab s + df(\nab s) +f \nab^{2} s
\\ = f\nab^2 s$$
\end{example}
We now want to connect this $\AAA(M)$- linear map with the earlier curvature tensor
\begin{proposition}
 The vector bundle map $\nab^2$ is equal to the curvature tensor $\Theta$
\end{proposition}
\begin{proof}
 We can work in local coordinates, and assume that X,Y are the standard commuting vector fields $\partial_i$ ,$\partial_j$.
We want to show that, given a section s, we have 
\\$$\nab^2(s)(\partial_i , \partial_j) = (\nab_{\partial_i} \nab_{\partial_i} - \nab_{\partial_j}\nab_{\partial_i})s 
\in E(M)$$
To do this we should check how $\nab$ was defined. Namely we have, by definition 
$\nab s = \sum_i dx_i \nab_{\partial_i} s$
and consequently 
$$\nab^2 s = \sum_{ij} dx_j \nab_{\partial_j}(dx_i \nab_{\partial_i}s)$$
This becomes, by the sign rules 
$\sum_{i\lt j}(\nab_{\partial_j \nab_{\partial_i} -  \nab_{\partial_i} \nab_{\partial_j}}s dx_i \wedge dx_j$.
It is easy to see that this, evaluated on on $(\partial_i,\partial_j)$, gives the desired quantity.
It follows that $\nab^2$ is equal to the curvature tensor $\Theta$
\end{proof}
As a result, we may calculate {\bf curvature} in a frame. Let $\Fff = {e_1,\cdots,e_n}$ be a frame and let $\theta(\Fff)$
be the connection matrix. Then we can obtain an n-by-n {\bf curvature matrix} $\Theta(\Fff)$ of 2-forms such that
\\$$ \Theta(\Fff) = \nab^2(\Fff)$$
The follow result enables us to compute $\Theta(\Fff)$.
\begin{proposition}[Cartan]
\begin{equation}\Theta(\Fff)=d\theta(\Fff) - \theta(\Fff)\wedge \theta(\Fff)\end{equation}
Note that $\theta(\Fff) \wedge \theta(\Fff)$ is not zero in general! The reason is that one is working with matrices of 1-forms,
not just plain 1-forms. The wedge product is a matrix product in a sense.
\end{proposition}
\begin{proof}
 Indeed, we need to determine how $\nab^2$ acts on the fram ${e_i}.$ Namely with an abuse of notation:
$$\nab^2(\Fff)=\nab(\nab \Fff) = \nab (\theta \Fff)(\Fff) = d \theta(\Fff)\Fff - \theta(\Fff) \wedge (\theta(\Fff)\Fff)$$
We have used this formula that describes how $\nab$ acts on a product with a form. As a result the proof holds.\end{proof}
Finally, we shall need an expression for $d \Theta$. We state this in terms of a local frame.
\begin{proposition}[Bianchi identity]
 With respect to a frame $\Fff$
$d \Theta(\Fff) = [\theta(\Fff),\Theta(\Fff)]$

\end{proposition}
Here the right side consists of matrices, so we talk about the commutator. We shall use this identity at a crucial point in showing
that the Chern-Weil homomorphism is even well-defined.
\begin{proof}
 This is a simple condition. For, by Cartan's equations,
$$d\Theta(\Fff) = d(d\theta(\Fff)- \theta(\Fff) \wedge \theta(\Fff)$$
= $$-d\theta(\Fff)\wedge \theta(\Fff) + \theta(\Fff) \wedge d\theta(\Fff)$$
Similarly,
\\ $$[\theta(\Fff),\Theta(\Fff)]=[\theta(\Fff),d\theta(\Fff)+\theta(\Fff)\wedge \theta (\Fff) \wedge \theta (\Fff)]$$
\\ $[\theta(\Fff),d\theta (\Fff)]$ because $[\theta(\Fff),\theta(\Fff)\wedge \theta(\Fff)] =0$
\end{proof}
