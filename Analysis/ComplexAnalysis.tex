\section{Complex Analysis}
When studying for a Riemann Surfaces class, I felt it was necessary to write down a few of the 'classical' Complex
Analysis definitions. It is beneficial (for reference) to have these all in a short document.
\subsection{Some definitions}
Singularities are very interesting, the following largely comes from the excellent textbook 
\cite{Shaw_CA_with_Mathematica} 

A function can fail to be holomorphic in a variety of ways. A large class of interesting cases can be managed
by exploring those singularities that are isolated in the same sense that zeroes are necessarily isolated. This
excludes cases such as Log(z) or $\sqrt{z}$ near the origin, where we have to introduce branch cuts just to have
a well-defined function, but does include a large and very important set of possibilities. 
\begin{dfn}[Definition of an isolated singularity]
 Suppose that we have an open U $\subset \Cbb$ and that $z \in \Cbb$. We define $U^* = U - {z}$. We say that a 
function f that is holomorphic on $U^*$ has an \textit{isolated singularity} at z. Since U is open, $N_{r}(z) \subset
U$ for some r, and  we define $N_{r}^*(z) = N_r(z) \cap U^*$. This is just a punctured disk, which is a special
case of an annulus, so we can write down a Laurent expansion:
\begin{equation}
 f(z) = \sum_{n=-\infty}^{-1} a_n(z-z_{0})^{n} + \sum_{n=0}^{\infty} a_{n} (z-z_{0})^{n}
\end{equation}

\end{dfn}
\textbf{Classification of isolated singularities}
The Laurent series allows us to classify isolated singularities into three types:
\begin{enumerate}
 \item The point z is said to be a \textit{removable singularity} if $a_{-n} = 0$ for all $n \gt 0$.
\item If $a_{-N}\neq 0$ but $a_{-n}=0$ for all $n \gt N$ then z is a \textit{pole of order} N. 
\item If infinitely many negative terms are present, then z is an \textit{isolated essential singularity.}
\end{enumerate}
We also have a very important removable singularity theorem.
\begin{thm}[Riemann removable singularities theorem]
If z is an isolated singularity of f(z) then it is a removable singularity if and only if there is an r $\gt 0$ such that
$f(z)$ is bounded on $N^*_{r}(z)$.
 
\end{thm}
\begin{rem}
 Note that one way is obvious - if the function indeed has a Taylor series it is bounded on a neighbourhood of z.
For the other way, consider the negative terms in the Laurent series: 
\begin{equation}
 a_{-n} = \dfrac{1}{2\pi i} \int_{\phi_{r}} f(z) (z-z_{0})^{n-1} dz
\end{equation}
 Now f is bounded on a circle centred on $z_{0}$ of radius s, for any s with $0 \lt s \lt r$, so 
suppose that $|f| \lt M$ for $|z| \lt r.$
\begin{equation}
 |a_{-n}| \leq \dfrac{1}{2\pi}M s^{n-1} 2 \pi s = Ms^{n}
\end{equation}
Since s can be as small as we will with $a_{-n} = 0$
\end{rem}
\begin{thm}[Identity Theorem]
 Suppose X and Y are Riemann surfaces and $f_{1}, f_{2}: X \to Y$ are two holomorphic mappings which coincide on a
set $A \subset X$ having a limit point $a \in X$. Then $f_1$ and $f_2$ are identically equal. 
\end{thm}
\begin{proof}
 Let G be the set of all points $x \in X$ having an open neighbourhood W such that $f_1|W= f_2|W$. By definition 
G is open. We claim that G is also closed. For, suppose b is a boundary point of G. Then $f_1(b) = f_2(b)$ since 
$f_1$ and $f_2$ are continuous. Choose charts $\phi: U \to V$ on X and $\phi:U' \to V'$ on Y with $b \in U$ and 
$f_1(U) \subset U'$. We may also assume that U is connected. The mappings
\begin{equation}
 g_{i} := \psi \circ f_i \circ \phi^{-1} : V \to V' \subset \Cbb
\end{equation}
are holomorphic. Since $U \cap G \neq \emptyset$, the Identity Theorem for holomorphic functions in \Cbb \; implies $g_1$
and $g_2$ are identically equal. Thus $f_1|U = f_2|U$. Hence $b\in G$ and thus G is closed. Now since X is connected
either G = $\emptyset$ or G = X. But the first case is exluded since $a \in G$ (using the Identity Theorem in the plane 
again). Hence $f_1$ and $f_2$ coincid on all of X. \qed
\end{proof} 
\begin{dfn} Let X be a Riemann surface and Y be an open subset of X. 
 By a \textit{meromorphic function} on Y we mean a holomorphic function $f:Y' \to \Cbb$, 
where $Y' \subset Y$ is an open subset, such that the following hold:
\begin{itemize}
 \item $Y\ Y'$ contain only isolated points
\item For every point $p \in Y\ Y'$ one has 
\begin{equation}
 \lim_{x \to p} |f(x)| = \infty.
\end{equation}
The points of $Y \ Y'$ are called the \textit{poles} of f. The set of all meromorphic functions on Y is denoted by 
$\mathcal{M}(Y)$.
\end{itemize}

\end{dfn}
\begin{rem}
 (a)Let (U,z) be a coordinate neighbourhood of a pole p of f with $z(p) = 0.$ Then f may be expanded in a Laurent
series 
\begin{equation}
 f = \sum_{v=-k}^{\infty} c_{v}z^{v}
\end{equation}
in a neighbourhood of p.
(b) $\mathcal{M}(Y)$ has the natural structure of a $\Cbb$-algebra. First of all the sm and the product of two meromorphic 
functions $f,g \in \mathcal{M}(Y)$ are holomorphic functions at those points where both f and g are holomorphic. Then 
one holomorphically extends, using Riemann's Removable Singularities Theorem, f+g (resp. fg) across any
singularities which are removable.
\end{rem}
