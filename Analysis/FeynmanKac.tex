\section{Some Probabilistic formula for the solutions of PDEs}
One of the more modern treatments for solving PDEs involves Stochastic Calculus. This can in turn be part of the 
larger field of 'Stochastic Geometry', however that would take us too far afield. 
Let us continue with the examples
\subsection{Feynman Kac}
Let L be a 2nd Order PDO on M, 
e.g. a) M differentiable manifold and L = $A_0 + \dfrac{1}{2} \sum_{i=1}^{r} A^{2}_{i}$
b) $M = \Rbb^n$ and L = $\sum b_i \partial_i +
 \dfrac{1}{2} \sum_{i,j=1}^{n} (\sigma \sigma^{*})_{ij} \partial_i \partial_j$
For x $\in$ M, let $X_t(x)$ be an L-diffusion, starting from x at time t=0, i.e. $X_{0}(x) = x$.
Note that $X_t(x)$ can be constructed as a solution of the SDE
\begin{equation}\label{SDE_1}
 dX = A_0 (X) dt + \sum_{i=1}^{r}A_{i}(X)\circ dW^{i}
\end{equation}
$X_0 = x$
\begin{equation}\label{SDE_2}
 dX = b(X) dt + \sigma(X) dW
\end{equation}
$X_0 = x$
where W is a Brownian Motion over $\Rbb^{r}$
Suppose that the lifetime of $X_t(x)$ is finite a.s. $\forall x \in M$.
\begin{prop}[Feynman-Kac formula]
Let $f:M \to \Rbb$ be continuous and bounded
    $V: M \to \Rbb$ be continuous and bounded above, i.e. $V(x) \leq K$ for K $\in \Rbb$
 Let u: $\Rbb_+ \times M \to \Rbb$ be a solution of the folllowing \textbf{initial value problem}(IVP) and note that
u needs to be bounded. 
\begin{align}
 \frac{\partial}{\partial t} u = Lu + Vu
\\ u|_{t=0} = f
\end{align}
i.e 
\begin{equation}
  (\frac{\partial}{\partial u}) (t_i) = Lu(t_i) + V(\cdot)u(t_i)
\end{equation}
$u(t, \cdot) = f$ on M
Then u is given by the following formula 
$$u(t,x) = \mathbb{E}[exp(\int_{0}^{t}V(X_{s}(x))ds)\cdot f(X_{t}(x))]$$

\end{prop}

\begin{rem}
 A remark which is unimportant for this course, is that there are substantial and deep links between $u(t,\cdot)= 
e^{tH}f$- and semigroup theory. 
\end{rem}
\begin{proof}
 Fix t $>$0. Consider 
\begin{equation}
 Y_{s}:= A_s \cdot Z_s
\end{equation}
where $A_S = \exp(\int_{0}^{s} V(X_r(x))dr)$and $Z_s:= u(t-s,X_s(x))$ for $0 \leq s \leq t$
Then $(Y_s)_{0\leq s \leq t}$ is a Martingale
\textbf{Indeed}
\end{proof}
\subsection{Schrodinger Operators}
\begin{rem}
 Operators of the form L+V = H where (V is the multiplication operator by V, or potential) and L is a Laplacian, and H
is the Hamiltonian are called \textbf{Schrodinger Operators}.
\end{rem}
We start with a definition
\begin{dfn}
Let $V: \Rbb^n \to \Rbb$ be a real-valued function.
The \emph{Schr\"odinger operator} \textbf{H} on the Hilbert space $L^2(\Rbb^n)$ is given by the action
\[
\psi \Rightarrow -\nab^2\psi+V(x)\psi, \quad\psi\in L^2(\Rbb^n).
\]

This can be obviously re-written as:

\[
\psi \Rightarrow [-\nab^2 +V(x)]\psi, \quad\psi\in L^2(\Rbb^n),
\] where $[-\nabla^2 +V(x)]$ is the {\em Schr\"odinger} operator, which is now
called the Hamiltonian Operator, \textbf{H}.
\end{dfn}
For stationary quantum systems such as electrons in `stable' atoms the {\em Schr\"odinger equation}
takes the very simple form :
\[
\textbf{H} \psi=E \psi
\] , where $E$ stands for energy eigenvalues of the stationary quantum states. Thus, in quantum mechanics of systems with finite degrees of freedom that are `stationary', the Schr\"odinger operator is used to calculate the (time-independent) energy states of a quantum system with potential energy $V(x)$. 
Schr\"odinger called this operator the Hamilton operator, or the
Hamiltonian, and the latter name is currently used in almost all of quantum physics publications, etc. 
The eigenvalues give the energy levels, and the wavefunctions are given by the eigenfunctions.
In the more general, non-stationary, or `dynamic' case, the Schr\"odinger equation takes the general form:

\[
\textbf{H} \psi= (-i) \frac{\partial \psi}{\partial t}
\]. 