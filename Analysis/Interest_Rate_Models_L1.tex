\section{Introduction}

The final exam is a 2 hour written exam which involves a combination of computation and the construction of 
models. 
In this course we will describe some of the main developments in interest-rate
modelling since Black and Scholes (1973) and Merton's (1973) original articles
on the pricing of equity derivatives. In particular, we will focus on continuous-
time, arbitrage-free models for the full term structure of interest rates. Other
models which model a limited number of key interest rates or which operate in
discrete time (for example, the Wilkie (1995) model) will be considered elsewhere.
Here we will describe the basic principles of arbitrage-free pricing and cover var-
ious frameworks for modelling: short-rate models (for example, Vasicek, Cox-
Ingersoll-Ross, Hull-White); the Heath-Jarrow-Morton approach for modelling
the forward-rate curve; and finally market models.
The course works through various approaches and models in a historical sequence.
Partly this is for history's sake, but, more importantly, the older models are
simpler and easier to understand. This will allow us to build up gradually to the
more up to date, but more complex, modelling techniques.
The definitive graduate textbook for this course is \cite{filipovic2009term}
There are plenty of other references including the books by the following academics 
and academic-neophytes\cite{TSM_Gibson_1999,IRM_Cairns, 
IRM_Brigo_2001,
FinCalc1996_Baxter,InterestRateDynamics}
or for a more 'practical' approach \cite{TermStructureModels} and an extensive review of Term Structure Models 
is \cite{Riccardo_TSM_Review} or the book chapter \cite{ModernRiskManagement}
\section{Preliminary remarks} 
\begin{dfn}
 Interest: Fee paid by the borrower to the lender, for having borrowed money over a certain period of time.
(One says the lender invests the money)
Interest rates are computed by means of:
\begin{enumerate}
 \item A nominal value M, expressed by \textlira,\pounds \;, \textdollar
\item A time length $\Delta$ expressed in some time unit: days, years, etc
\item A positive parameter R $\gt$ 0, the 'interest rates' 
\end{enumerate}

\end{dfn}
\begin{rem}
 TFAE: 

\begin{itemize}
 \item A future value M, invested over some time unit M'\pounds \; > M 
\item The interest rate generated by M invested over $\Delta$ is (M'-M) $\gt$ 0 at rate R.
\end{itemize}

\end{rem}
\begin{dfn}[Different Systems]
\begin{itemize}
 \item The ratio R is \textit{simple}, if $\Delta \leq$ 1 and the future value of M is invested over $\Delta$ is 
FV = M(1 + R $\Delta$), where the interest is MR$\Delta$
\item The rate R is 'complicated', if $\Delta$ = 1,2,3,..., hen the future value of M\pounds \;
 over $\Delta$ is 
$FV = M(1 + R)^{\Delta}$ 
\end{itemize}

\end{dfn}
\begin{rem}
 The \textbf{actual value} of M\pounds \; at t is T (t \lt T), associated with the short rate r is \textit{present value}
$PV = Me^{-r(T-t)}$ 
\end{rem}
\section{Building models} 
We want to build a model of a financial market in continuous time, with emphasis on fixed-income products (Financial
products providing deterministic cashflow). 
\textbf{Assumption H:}
\begin{enumerate}
 \item No transaction cost
\item Financial products are infinitely divisible
\item Short selling is available 
\item No liquidity risk 
\item Continuous time
\item Prices are linear
\end{enumerate}
\begin{dfn}
\begin{enumerate}
 \item A portfolio $\pi$ is a combination of financial assets held by a single investor. 
\item The value of a portfolio $\pi$ at time t is $\pi(t) =$ the sum of the value of components in t
\item We say that the market satisfies a static \textbf{No Arbitrage} (N.A.) property, if $\forall \pi, \pi^{'}$, the
following are held 
\begin{equation}\label{n.a}
 \pi(T) \leq \pi^{'}(T) \to \forall t \leq T, \pi(t) \leq \pi^{'}(t)
\end{equation}

\end{enumerate}

\end{dfn}
We call \ref{n.a} the \textit{condition to verify N.A.} 
\subsection{Zero Coupon bonds and Associated Interest rates} 
Before we begin with definitions of Zero coupon bonds, it is worth us including some of the nomeclature from the
financial industry. Since some of these definitions are easily forgotten!
\begin{dfn}
\begin{itemize}
 \item spot market: \textit{immediately} exercised trades (notice value 'date').
 \item fixed income: interest rate (IR) trading.
 \item money market: IR products with maturity in \pounds \;1y.
 \item bond market: maturity \gt 1y (government bonds, corporate bonds).
\item swap market: interbank IR trading \gt 1y (has \textit{spot market qualities}).
\item  discount product: IR instrument, traded at a rebate w.r.t. its nominal
value (discount bond, zero coupon bond, zero bond).
\item  coupon bond: instrument with periodic interest payments (coupons).
\item unconditional exerc.: the trade must be completed by both counterparties
in any case (forwards, futures).
\item conditional exercise: the option holder has the right to chose whether (or,
when) to exercise the trade, the option writer must
comply (option contracts; but notice generalised
contingent claims).
\item derivative: general term for financial instruments which are derived from
underlying (usually simpler) financial instruments.
\item underlying: short for underlying instrument of a derivative instrument.
\item maturity: calendar date on which a forward (or futures, or optional)
trade must (or can) be exercised, or a coupon (or the
principal) of a bond must be paid.
\item spot price: current market value of an asset (underlying instrument).
\item forward price: currently determined price to be paid (by the long c/party) in
a forward contract for delivery (by the short counterparty) of
the underlying asset at maturity (known today).
\item futures price: dto. for a futures contract (known today).
(Usually very close or equal to forward price.)
\item future price: actual spot price of the asset in the future (unknown today).
\item  payoff: value of a derivative contract at maturity.
\item  Forward contracts: difference between forward price at conclusion of
contract and spot price at maturity
\end{itemize}
There are two major types of traded financial products, those over the counter and those 
traded on the various financial exchanges around the world. 

\textbf{exchange traded product}: financial instrument offered by a futures or options
exchange which acts as intermediary to all counterparties
\begin{itemize}
\item standardised contract specifications
 no individual negotiation of product features
\item margin account system, low counterparty risk
\item usually highly liquid trading
\item innovation possible
\item example: futures contract
\end{itemize}
\textbf{OTC product}: a trade where two c/parties deal directly without intermediary.
\begin{itemize}
\item no strict standardised features, but standard conventions exist (for some markets)
\item all contract specifications negotiated individually
\item  counterparty risk depends on (the credit rating of) your counterparty
\item liquidity ranges from very high (FX spot and derivatives trading) to very low
(structured/complex deals)
\item highly innovative products
\item  example: forward contract
\end{itemize}
\end{dfn}

\begin{dfn}
 A \textbf{forward contract} is an agreement between two entities (counterparties) which
commits both buyer (long position) and seller (short position) at time t ('today'),
to buy resp. to sell a particular object S (underlying) at a certain future time
(maturity) T $\gt$ t at a predetermined price K (forward price).
\end{dfn}

\begin{rem}
\begin{itemize}
 \item The Gain for the long position in the contract equals the loss for the short
position and vice versa (zero sum game).
\item The deal must be 'fair' to both parties. This is achieved by choosing the
forward price K such that the contract has no value at its conclusion. Our
task is the determination of the 'fair' forward price K.
\end{itemize}
\end{rem}

\begin{dfn}
 A zero bond is a financial instrument without running interest payments and only
a single cash flow at maturity. Zero bonds are issued (and traded) below their
nominal value (redemption value).
\end{dfn}
Or alternatively, and equivalently
\begin{dfn}
Let T \gt 0, a zero coupon bond with the maturity  is a contract guarranteeing the owner 
$1 \pounds$ at time T. 
\end{dfn}
\begin{dfn}A coupon bond is a financial instrument which pays periodic interest, plus
once at maturity the principal amount (nominal value).
\end{dfn}
Let us introduce some notation to handle Zero-coupon bounds 
\begin{dfn}
 Let P be a zero-coupon bound with the maturity T. 
$P(t,T)$ = the price of P at time $t \lt T$ value. Plainly $P(T,T) = 1$
\end{dfn}
We have some assumptions
\begin{rem} 
 \begin{enumerate}
  \item There exists a market which satisfies assumption H, where bonds of any maturity are sold and bought
\item $0 \leq P(t,T) \leq 1$
\item T $\to$ P(t,T) the \textit{term structure} of zero coupon bond prices. Smooth (T $\to$ P(t,T)) - for every
fixed T. Differentiable and continuous
 \end{enumerate}
\end{rem}
\subsection{Relation with Forward Rate Agreement (FRA)} 
\begin{dfn}
 A Forward Rate Agreement is an over-the-counter contract between two parties specificying an interest rate (fixed
in a future period) to be applied at a future date.
\end{dfn}
\begin{rem}
We can use Zero-coupon bonds to \textit{simulate} a FRA
For instance, consider dates $t \lt T \lt S$ 
We have the following strategy at t:sell one T-bond and buy $\frac{P(t,T)}{P(t,S)}$ s-bond
At T: pay 1$\pounds$ 
At S: Touch $\frac{P(t,T)}{P(t,S)}\cdot P(s,S)= \frac{P(t,T)}{P(t,S)}$ in S $\geq$ 1. 
Since $T \to P(t,T)$ is non-increasing
\end{rem}
