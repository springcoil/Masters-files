\section{Interest Rate Models - Lecture 3}
We continue with our investigation of 
There are versions of coupon bonds for which the value of the coupon is not fixed
at the time the bond is issued, but rather reset for every coupon period. Most often
the resetting is determined by some market interest rate (e.g. LIBOR).
A floating rate note is specified by:
\begin{itemize}
\item a number of future dates $T_0$ <$T_1$ < $\cdots$<$T_n$,
\item a nominal value N.
\end{itemize}
The deterministic coupon payments for the fixed coupon bond are now replaced by
$c_i$ = ($T_i$ − $T_i$−1)F ($T_i$−1,$T_i$)N,
where F($T_i$−1,$T_i$ ) is the prevailing simple market interest rate, and we note that
F($T_i$−1,$T_i$ ) is determined already at time $T_i$−1 (this is why here we have $T_0$ in
addition to the coupon dates $T_1$,$\cdots,$ $T_n$), but that the cash flow $c_i$ is at time $T_i$ .
The value p(t) of this note at time t  $ \leq T_0$ is obtained as follows. Without loss of
generality we set N = 1. By definition of F($T_i$−1,$T_i$ ) we then have
\begin{equation}c_i = 1
P(T_i−1,T_i) −1.\end{equation}
The time t value of −1 paid out at $T_i$ is −P(t,$T_i$ ). The time t value of 1
P($T_i$−1,$T_i$ )
paid out at $T_i$ is P(t,$T_i$−1):
\begin{itemize}
\item At t: buy a $T_i$−1-bond. Cost: P(t,$T_i$−1).
\item At $T_i$−1: receive one dollar and buy 1
P($T_i$−1,$T_i$ ) $T_i$-bonds. This is a zero net investment.
\item At $T_i$ : receive $\frac{1}
{P(T_i−1,T_i )}$dollars.
\end{itemize}
The time t value of $c_i$ therefore is
P(t,$T_i$−1)−P(t,$T_i$ ). (2.3)
Summing up we obtain the (surprisingly simple) formula
\begin{equation}
p(t) = P(t,T_n)+
\Sigma^{n}
_{i=1}
(P(t,T_i−1)−P(t,T_i )) = P(t,T_0).\end{equation}
In particular, for t = $T_0$, we obtain p($T_0$) = 1.
\begin{prop}
 Under no-arbitrage, $\forall t \leq T_0$ Price $\left\lbrace t;
 FRN \right\rbrace:= P_{FRN}(t) = N \times P(t,T_0)$
\end{prop}
\begin{proof}
 Without lost of generality, let us set N=1; Fix $i=1, \cdots, n $
Problem: We want to 'generate' in t the cashflow $c_i$
\begin{equation}
Strategy = \begin{cases}
   t \quad \text{ buy 1 }T_{i-1}\text{ bond}\\  
\text{ sell 1} T_{i} \text{ bond } P(t,T_{i-1}) - P(t,T_{i}) \geq 0\\
   T_i \quad \text{ receive } \dfrac{1}{P(T_{i-1}, T_{i}} \pounds, \text{ and pay }  \end{cases}
\end{equation}
\end{proof}

\subsection{Interest Rate Swaps}
An interest rate swap is a scheme where you exchange a payment stream at a fixed
rate of interest for a payment stream at a floating rate (typically LIBOR).
There are many versions of interest rate swaps. A payer interest rate swap settled
in arrears is specified by: