\section{Rapid review of the Stochastic Calculus}
It became obvious when reviewing the material for Interest rate models that the underlying Mathematics is actually
the Stochastic Calculus. So let us proceed. 
The \textbf{Stochastic basis} is a filtered probability space $(\Omega, \mathcal{F}, (\mathcal{F}_{t})_{t \geq 0}, \mathbb{P})$
satisfying the usual conditions \footnote{The usual conditions are (1) completeness: $\mathcal{F}_{0}$ contains all of the
null sets, and (2) right continuity: $\mathcal{F}_{t} = \cap_{s > t} \mathcal{F}$ for all $t \geq 0$}
and carrying a d-dimensional $(\mathcal{F}_{t})$-adapted BM $W = (W_1, \cdots, W_d)^{\intercal}$. We shall assume 
that $\mathcal{F} = \mathcal{F}_{\infty} = \cup_{t \geq 0} \mathcal{F}_{t}$, we do not a priori fix a finite time 
horizon. 
 We write $\mathfrak{B}[0,t]$ or simply $\mathfrak{B}$ for the respective Borel - $\sigma$-algebra. A Stochastic
process $X = X(\omega, t)$ is called: 
\begin{itemize}
 \item adapted if $\Omega \ni \omega \Rightarrow X(\omega,t)$ is $\mathcal{F}_{t}$ measurable $\forall t \geq 0$
\item progressively measurable (or simply progressive) if $\Omega \times [0,t] \ni (\omega, s) \Rightarrow X(\omega, s)$

\end{itemize}
A progressive process is obviously adapted. Progressive measurability of X is needed in order that composed
processes such as $\int_{0}^{t} X(s)ds$ and $X(min(t, \tau))$, for any stopping time $\tau$, are adapted. 
 \begin{dfn}
  Prog: the progressive $\sigma$-algebra, generated by all progressive process, on $\Omega \times \mathbb{R}_{+}$
 \end{dfn}
\subsection{Stochastic Integration} 
We now define $\mathcal{L}^{2}$ and $\mathcal{L}$ as the sets of all $\Rbb^{d}$-valued progressive proces 
h = ($h_1, \cdots, h_d$) that satisfy 
\begin{equation}
 \mathbb{E}[\int_{0}^{\infty} \|h(s) \|^{2} ds] \lt \infty
\end{equation}
and 
\begin{equation}
 \int_{0}^{t} \|h(s)\|^{2} ds \lt \infty \; \forall t \gt 0
\end{equation}
respectively. The inclusion $\mathcal{L}^{2} \subset \mathcal{L}$ is obvious.