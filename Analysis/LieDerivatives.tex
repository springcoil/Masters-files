\section{Lie Derivatives}
The aim is to give a nice contained introduction to the properties of the Lie Derivative. This is covered 
in any good book on analysis on manifolds, or differential geometry, such as \cite{jost2011riemannian,differentialmanifolds}
note that Warner\cite{differentialmanifolds} is better seen as a reference book rather than a book for learning from. 
Also one can see the 'Basic Riemannian Geometry' by FE Burstall as a chapter in \cite{davies1999spectral}.
A pdf is also available from \cite{basicriemanniangeometry}
The Lie Derivative is a method of computing the 'directional derivative' of a vector field with 
respect to another vector field. We defined the \newword{Lie Derivative of a function} $f \in C^{\infty}(M)$ in the
direction of a vector field X $\in Vect(M)$ by
\begin{equation}
 (L_X f)_m = X_m(df)_m
\end{equation}
for all $m \in M$. We have an isomorphism between $X \in Vect(M)$ and $L_X$ $\in Der(C^{\infty}(M))$ so we
can write $L_X$as X.
   We would like to extend these ideas to vector fields. On a manifold it isn't as easy as in a Euclidean space. The
problem is due to the fact different vectors live in different tangent spaces. So we replace the notion of vectors $X \in T_m M$
with that of a vector field. 
\begin{idea}
 We can now use the \textbf{flow} of the vector field to \textbf{push} values of Y back to m and differentiate.
\end{idea}
We say the Lie derivative of Y with respect to the vector field X
We have to compare the value $Y_m \in T_m M$ of Y at m with the value of Y at a point of M that lies close to m in
the direction specified by X, i.e. with the value 
\begin{equation*}
 Y_{\phi^X_t}(m) \in T_{\phi^{X}_t}(m)M
\end{equation*}
We want to use the flow of X to \textbf{push} the values of $Y_{\phi^X_t}(m)$ back to $Y_m$and then differentiate.
\begin{dfn}
 \begin{equation*}
  (L_X Y)_m = lim_{t\To0} \frac{(\phi^{X}_{-t,*}Y)_m-(\phi^{X}_{-0,*}Y)_m}{t}
 \end{equation*}
\begin{equation*}
 = \frac{d}{dt}|_{t=0} (\phi^{X}_{-t,*}Y)_m
\end{equation*}

\end{dfn}
Where $(\phi^{X}_{-t,*}$ denotes the pushforware by $(\phi^{X}_{-t}$, this does allows us to cmpare two vectors that live in
different tangent spaces. 