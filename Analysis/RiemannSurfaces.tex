\documentclass[12pt, oneside, a4paper]{article}

\usepackage[all]{xy}
\usepackage[english]{babel}
\usepackage[T1]{fontenc}
\usepackage{amsthm, amsmath, amssymb, amsfonts, color, hyperref}

\newcommand{\bb}[1]{\textbf{#1}}

\newtheorem{thm}{Theorem}[section]
\newtheorem{lem}[thm]{Lemma}
\newtheorem{prop}[thm]{Proposition}
\newtheorem{cor}[thm]{Corollary}
%\newtheorem{clm}[thm]{Claim}

\theoremstyle{definition}
\newtheorem{dfn}[thm]{Definition}
\newtheorem{rem}[thm]{Remark}
%\newtheorem*{remu}{Remark}
\newtheorem{ex}[thm]{Example}
\newtheorem{exs}[thm]{Examples}

\newtheorem{metaphor}{Metaphor}
\newtheorem{idea}{Idea}
\newtheorem{convention}{Convention}

\def \grad {\overrightarrow{\nabla}}
\def \curl {\overrightarrow{\nabla} \wedge \cdot}
\def \div {\overrightarrow{\nabla} \cdot}
\def \nab {\ensuremath{\nabla}}
\def \im {\text{im }}
\def \ker {\text{ker }}



\def \C {\mathcal{C}} % for categories
\def \D {\mathcal{D}}
\def \S {\mathcal{S}}
\def \P {\mathcal{P}}



\def\Cbb{\ensuremath{\mathbb{C}}}
\def\Kbb{\ensuremath{\mathbb{K}}}
\def\Nbb{\mathbb{N}}
\def\Pbb{\mathbb{P}}
\def\Qbb{\mathbb{Q}}
\def\Rbb{\ensuremath{\mathbb{R}}}

\def \eps {\varepsilon}
\def \AA{\ensuremath{\mathcal{A}^{1}}}
\def  \AAA{\ensuremath{\mathcal{A}}}
\def \Fff{\mathfrak{F}}
%\def \commutes {\ar@{}[rd]|{\circlearrowleft}}
\def \commutes {\ar@{}[rd]|{\mbox{ \Large{$\circlearrowleft$} }}}

\newcommand{\tensor}{\otimes}
\newcommand{\gt}{>}
\newcommand{\lt}{<}
\newcommand{\itexarray}[1]{\begin{matrix}#1\end{matrix}} %To draw commutative Diagrams
\newcommand{\Ob}{{\rm Ob}}
     \newcommand{\Cob}{{\rm Cob}}   
	\newcommand{\Vect}{{\mathrm{Vect}}}   
	\newcommand{\Hilb}{{\mathrm{Hilb}}}    
	\newcommand{\tr}{{\rm tr}}   
      \newcommand{\Hom}{{\rm hom}}
\newcommand{\cChVect}{\mathrm {cCh(Vect)}}
\newcommand{\gVect}{\mathrm {gVect}} 
\newcommand{\To}{\Rightarrow}  
\newcommand{\newword}[1]{\textbf{\emph{#1}}}
\title{\bf Riemann Surfaces}

\author{Peadar Coyle}

\begin{document}

\maketitle

\newpage

\tableofcontents

\newpage
\section{Answers for Riemann Surfaces}
\begin{ex}
Proof of the \Cbb-algebra structure of meromorphic functions
\begin{proof}
 $\mathcal{M}(Y)$ has the natural structure of a $\Cbb$-algebra. First of all the sm and the product of two meromorphic 
functions $f,g \in \mathcal{M}(Y)$ are holomorphic functions at those points where both f and g are holomorphic. Then 
one holomorphically extends, using Riemann's Removable Singularities Theorem, f+g (resp. fg) across any
singularities which are removable.
\end{proof}
\end{ex}

\begin{ex}\textbf{Exercise 6}
2) Let $\Gamma$ be a lattice in C. Can you describe all holomorphic functions on the torus $$\frac{\Cbb}{\Gamma}$$ using
a similar reasoning as in part 1) of this exercise?
\begin{proof}
Any holomorphic function on complex tori is constant, as it is a bounded holomorphic function on the 
universal covering space, and by Liouvilles theorem this is constant (and by the Maximum moduli theorem)
\end{proof}
\end{ex}
\begin{ex}
 \textbf{Exercise 9}
Let us consider a point $p\in P$, where P is the set of poles. 
We've defined
$\hat{f}(p) = f(p)$ if p $\notin$P 


Alternatively the function $\hat{f}$ is infinity. 
But we know that p is definitely in the set of poles, and we know that $\hat{f}$ is mapping from the domain $\infty$
to the codomain $\infty$. On the Riemann sphere, by definition, this is therefore a continuous map. 

\end{ex}
\begin{ex}\textbf{Exercise 12}
1) Let $$f : X \rightarrow Y$$ be a non-constant morphism of Riemann surfaces. Prove
that if X is compact, then f is surjective. In particular, Y is compact as well in this case.
2) Use the first part of this exercise to conclude that $\mathcal{O}_X(X) = \mathbb{C}$ for every compact Riemann
surface X, i. e., the only holomorphic functions on a compact Riemann surface are constants.

Let us introduce a corollary
\begin{cor} Let $f : M \rightarrow N$ be a non-constant holomorphic mapping, then f is open,
i.e. an image of any open set is open.
\end{cor}
\begin{cor} Let $f : M \rightarrow N$ be a non-constant holomorphic mapping and M compact.
Then f is surjective f(M) = N and N is also compact.
\end{cor}
\begin{proof} The previous corollary implies that f(M) is open. On the other hand, f(M)
is compact since it is a continuous image of a compact set. f(M) is open, closed and
non-empty, therefore f(M) = N and N compact.
\end{proof}
For question 2, it is simply a case of Liouvilles theorem. 
\begin{thm}[Liouville]
There are no non-constant holomorphic functions on compact Riemann surfaces.
\begin{proof} An existence of a non-constant holomorphic mapping $f : M \rightarrow \mathbb{C}$ contradicts to
the previous corollary since \Cbb is not compact.
\end{proof}
\end{thm}

\end{ex}
\begin{ex}\textbf{Exercise 14}

 \begin{prop}Any doubly periodic holomorphic function on \Cbb is constant.\end{prop}
\begin{proof}: Global holomorphic functions on $\dfrac{\Cbb}{L}$ are constant.\end{proof}
\begin{proof} By Liouville's theorem, bounded holomorphic functions on \Cbb  are constant. \end{proof}
To get any interesting functions, we must allow poles.
\begin{dfn}: An elliptic function is a doubly periodic meromorphic function on \Cbb.\end{dfn}
Elliptic functions are thus meromorphic functions on a torus $\dfrac{\Cbb}{L}$. The reason for the name is
lost in the dawn of time. (Really, elliptic functions can be used to express the arc-length on the
ellipse.)
\end{ex}
\begin{ex}\textbf{Exercise 15}
Prove that a Riemann surface X is path-connected.
\begin{proof}
 Consider a point $x_{0}$ and consider the set of all points S which $x_{0}$ can be connected to by a path.
This set S is an open subset of X, and all open subsets of X can be mapped to an open subset of \Cbb which is
open, closed and non-empty.
\end{proof}

 
\end{ex}
\begin{ex}
 Prove that 0 is the only ramification point of the holomorphic map 
$\Cbb \to \Cbb$, $z \mapsto z^{k}$, $k \geq 2$
\end{ex}

\newpage


\end{document} 
