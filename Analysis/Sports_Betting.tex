\documentclass[12pt, oneside, a4paper]{article}

\usepackage[all]{xy}
\usepackage[english]{babel}
\usepackage[T1]{fontenc}
\usepackage{amsthm, amsmath, amssymb, amsfonts, color, hyperref}

\newcommand{\bb}[1]{\textbf{#1}}

\newtheorem{thm}{Theorem}[section]
\newtheorem{lem}[thm]{Lemma}
\newtheorem{prop}[thm]{Proposition}
\newtheorem{cor}[thm]{Corollary}
%\newtheorem{clm}[thm]{Claim}

\theoremstyle{definition}
\newtheorem{dfn}[thm]{Definition}
\newtheorem{rem}[thm]{Remark}
%\newtheorem*{remu}{Remark}
\newtheorem{ex}[thm]{Example}
\newtheorem{exs}[thm]{Examples}

\newtheorem{metaphor}{Metaphor}
\newtheorem{idea}{Idea}
\newtheorem{convention}{Convention}

\def \grad {\overrightarrow{\nabla}}
\def \curl {\overrightarrow{\nabla} \wedge \cdot}
\def \div {\overrightarrow{\nabla} \cdot}
\def \nab {\ensuremath{\nabla}}
\def \im {\text{im }}
\def \ker {\text{ker }}



\def \C {\mathcal{C}} % for categories
\def \D {\mathcal{D}}
\def \S {\mathcal{S}}
\def \P {\mathcal{P}}



\def\Cbb{\ensuremath{\mathbb{C}}}
\def\Kbb{\ensuremath{\mathbb{K}}}
\def\Nbb{\mathbb{N}}
\def\Pbb{\mathbb{P}}
\def\Qbb{\mathbb{Q}}
\def\Rbb{\ensuremath{\mathbb{R}}}

\def \eps {\varepsilon}
\def \AA{\ensuremath{\mathcal{A}^{1}}}
\def  \AAA{\ensuremath{\mathcal{A}}}
\def \Fff{\mathfrak{F}}
%\def \commutes {\ar@{}[rd]|{\circlearrowleft}}
\def \commutes {\ar@{}[rd]|{\mbox{ \Large{$\circlearrowleft$} }}}

\newcommand{\tensor}{\otimes}
\newcommand{\gt}{>}
\newcommand{\lt}{<}
\newcommand{\itexarray}[1]{\begin{matrix}#1\end{matrix}} %To draw commutative Diagrams
\newcommand{\Ob}{{\rm Ob}}
     \newcommand{\Cob}{{\rm Cob}}   
	\newcommand{\Vect}{{\mathrm{Vect}}}   
	\newcommand{\Hilb}{{\mathrm{Hilb}}}    
	\newcommand{\tr}{{\rm tr}}   
      \newcommand{\Hom}{{\rm hom}}
\newcommand{\cChVect}{\mathrm {cCh(Vect)}}
\newcommand{\gVect}{\mathrm {gVect}} 
\newcommand{\To}{\Rightarrow}  
\newcommand{\newword}[1]{\textbf{\emph{#1}}}
\title{\bf On Valuing Soccer Spread Bets}

\author{Peadar Coyle}

\begin{document}

\maketitle
\section{Spread betting}
To help me understand how the Mathematics of Spread betting works I've been working through a paper 
called \textit{The Valuation of Soccer Spread Bets} by A.D.Fitt, C.J.Howls et al.
Viewed from a Financial Mathematics point of view, a spread bet is simply a future (forward) contract. Share prices
, follow a lognormal random walk. 
How does one quantify the behaviour of sports games? 
\subsection{Multicorners}
Thinking about multicorners is fun. It also is (perhaps) some sort of arbitrage opportunity. 
To value the multicorner spread bet, we let X denote the product of the first and second half total corner counts and use 
$N(a,b)$ to denote the total number of corners taken during the time period $(a,b)$. Evidently X = $N(0,\frac{1}{2})
N(\frac{1}{2}, 1)$ and $E_{0}(X)= E(N(0,\frac{1}{2}))E(
N(\frac{1}{2}, 1))$. Denoting the Poisson mean of the total number of match corners by $\mu$ we can deduce that $E_{0}(X)
= \frac{\mu^{2}}{4}$


\end{document} 
