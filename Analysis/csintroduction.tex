\section{Introduction}
One of the active areas of geometric topology at the border to mathematical physics is the theory of 'Chern-Simons' theory.
The aim is to elucidate some of the techniques aimed at graduate students, and beginning researchers who know the basics
of differential forms and differential manifolds. 
 After an introduction to Lie groups and Lie algebras, principal bundles, connections and gauge transformations,
 we will carefully construct the Chern-Simons action and study the moduli space of its classical solutions. 
 This yields Taubes' beautiful and influential description of Casson's invariant for homology 3-spheres via
 Chern-Simons Theory for which we need such concepts as differential operators and spectral flow. This naturally
 leads to subjects like the eta invariant and the rho invariant on the one hand as well as the quantization of the
 Chern-Simons action and Witten's invariants on the other.
 Furthermore some modern examples from Theoretical Condensed Matter physics will be included. 
 \subsection{An overview}
 Let G be a semi-simple Lie group and $\mathfrak{g}$ its Lie algebra. A connection A on the trivial G-bundle over a 
 closed oriented 3-manifold M is a $\mathfrak{g}$-valued 1-form, i.e. $A \in \Omega^1(M;\mathfrak{g})$.
 Chern-Simons theory is a quantum field theory in three dimensions, whose action is proportional to the Chern-Simons
 invariant given in \cite{Chern1974}
    The celebrated work by Witten on the Jones Polynomial \cite{Witten:1988hf}.
The other work \cite{donaldson1997geometry,Tong:2005un} are also good references for some of the material referenced.
\begin{equation*}
 cs(A_ = \dfrac{1}{8\pi^{2}} \int_M tr(A \wedge dA + \dfrac{1}{3}A \wedge [A \wedge A]).
\end{equation*}
Witten introduced Chern-Simons theory to knot theory in 1989 \cite{Witten:1988hf}, when he described for each integer 
level $k \in \mathbb{Z}$ an invariant of a link $L = (L_{j})$ in a 3-manifold M (and a list of finite-dimensional 
representations $\rho_k$ of G associated to the link invariant $L_j$) as the (non-rigorous) Feynman path integral
\begin{equation*}
 Z_k(M,L) = \int_{\mathcal{A}/\mathcal{G}} \exp^{2\pi ki cs(A)} \prod_{j} tr_{\rho_{j}} (hol_A(L_j))dA,
\end{equation*}
where $\mathcal{A}$ is the space of G-connections, $\mathcal{G}$ is the space of gauge transformations. He interpreted
these invariants using the axioms of Topological Quantum Field Theory (TQFT) as well as via an asymptotic expansion
- the semiclassical approximation - by using the method of stationary phase. 
There have also been overlaps with knot Floer homology, and instantons, etc.
 
