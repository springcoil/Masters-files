\section{Topological Phases of Matter and Non-Abelian Anyons}
A useful review article is \cite{2008RvMP} by Freedman, Simon et al.
    Let us consider some of the theory of Topological Quantum Computation.
 Topological quantum computation is predicated on the existence in nature of topological phases of matter. In this section,
 we will discuss the physics of topological phases from several different perspectives, using a variety of theoretical tools.
 Topological phases, the states of matter which support anyons, occur in many-particle physical systems.
 Therefore, we will be using field theory techniques to study these states. A canonical, but by no means unique, example of a field
 theory for a topological phase is Chern-Simons theory. 
 \subsection{Chern-Simons Theory}
 Consider the simplest example of a TQFT, Abelian Chern-Simons theory, which is relevant to the Laughlin states at filling
 fractions of the form $\nu = \dfrac{1}{k}$, with k an odd interger. Although there are many ways to understand the Laughlin
 states, it is useful for us to take the viewpoint of a low-energy effective theory. Since quantum Hall systems are gapped, we
 should be able to describe the system by a field theory with very few degrees of freedom.
 Let us consider the action 
 \begin{equation}\label{19}
  S_{CS} = \dfrac{k}{4\pi} \int d^2 \mathbf{r} dt \epsilon^{\mu \nu \rho}a_{\mu}\partial_{\nu} a_{\rho}
 \end{equation}
where k is an integer $\epsilon$ is the antisymmetric tensor. Here $\alpha$ is a U(1) gauge field and indices $\mu, \nu, \rho$
atake the values 0 (for time direction), 1,2 (space-directions). The action represents the low-energy degrees of freedom of 
the system, which are purely topological.  \paragraph{}
   The Chern-Simons gauge field $\alpha$ in (\ref{19}) is an emergent degree of freedom which encodes the low-energy physics of a 
   quantum Hall system. Although in this particular cases, it is simple-related to the electronic charge density, we will also 
   be considering systems in which emergent Chern-Simons tauge fields cannot be related in a simple way to the underlying 
   electronic degrees of freedom. 
   \paragraph{} In the presence of an external electromagnetic field and quasiparticles, the action takes the form:
   \begin{equation}\label{20}
    S = S_{CS} - \int d^2\mathbf{r}dt \left(\dfrac{1}{2\pi}\epsilon^{\mu\nu\rho}A_{\mu}\partial_{\nu}a_{\rho} + j_{\mu}^{qp}a_{\mu} \right)
   \end{equation}
where $j^{op}_{\mu}$ is the quasiparticle current, $j_{0}^{qp}= \rho^{qp}$ is the quasi-particle density $\mathbf{j}^{qp} = 
(j_{1}^{qp}, j_{2}^{qp})$ is the quasiparticle spatial current, and $A_{\mu}$ is the external electromagnetic field. We will
assume that the quasiparticles are not dynamical, but instead move along some classically-prescribed trajectories which
determine $j_{\mu}^{qp}$. The electircal current is:
\begin{equation}\label{21}
 j_{\mu} = \dfrac{\partial\mathcal{L}}{\partial A_{\mu}} = \dfrac{1}{2\pi} \epsilon^{\mu \nu \rho}\partial_{\nu}a_{\rho}
\end{equation}
Since the action is quadratic, it is completely solvable, and one can integrate out the field $a_{\mu}$ to obtain the response
of the current to the external electromagnetic field\footnote{Why exactly does the action being quadratic mean that it is 
'completely solvable'?}.The result of such a calcuation is precisely the quantized Hall conductivity 
$\sigma_{xx} = 0$ and $\sigma_{xy} = \dfrac{1}{k} e^{2}/h.$
  \paragraph{} The equation of motion obtained by varying $a_{0}$ is the Chern-Simons constraint: 
  \begin{equation}\label{22}
   \dfrac{k}{2\pi}\nabla \times \mathbf{a} = j_{0}^{qp} + \dfrac{1}{2\pi} B
  \end{equation}
According to this equation, each quasiparticle has Chern-
Simons flux $2pi/k$ attached to it (the magnetic field is assumed
fixed). Consequently, it has electrical charge 1/k, accord-
ing to (\ref{21}). As a result of the Chern-Simons flux, another
quasiparticle moving in this Chern-Simons field picks up an
Aharonov-Bohm phase. The action associated with taking
one quasiparticle around another is, according to Eq.\ref{20}, of
the form 
\begin{equation}\label{23}\dfrac{1}{2} k \int d\mathbf{r}dt \mathbf{j} \cdot \mathbf{a} = kQ \int_{C} d\mathbf{r} \cdot \mathbf{a}\end{equation}
where Q is the charge of the quasiparticle and the final integral
is just the Chern-Simons flux enclosed in the path. (The factor
of 1/2 on the left-hand side is due to the action of the Chern-
Simons term itself which, according to the constraint (\ref{22}) is
-1/2 times the Aharanov-Bohm phase. This is cancelled by a factor of 
two coming from the fact that each particle sees the other's flux.) 
Thus the contribution to a path integral $e^{iS_{CS}}$
just gives an Aharonov-Bohm phase associated with moving
a charge around the Chern-Simons flux attached to the other
charges. The phases generated in this way give the quasipar-
ticles of this Chern-Simons theory $\theta = \pi/k$ Abelian braiding
statistics.\footnote{The Chern-Simons effective action for a hierarchical state is equivalent to the action for the composite fermion state at the same filling fraction.}
Therefore, an Abelian Chern-Simons term implements
Abelian anyonic statistics. In fact, it does nothing else. An
Abelian gauge field in 2 + 1 dimensions has only one trans-
verse component; the other two components can be eliminated
by fixing the gauge. This degree of freedom is fixed by the
Chern-Simons constraint (\ref{22}). Therefore, a Chern-Simons
gauge field has no local degrees of freedom and no dynam-
ics.
We now turn to non-Abelian Chern-Simons theory. This
TQFT describes non-Abelian anyons. It is analogous to the
Abelian Chern-Simons described above, but different meth-
ods are needed for its solution, as we describe in this section.
The action can be written on an arbitrary manifold $\mathcal{M}$ in the
form 
\begin{eqnarray}\label{24}
S_{CS}[a] = \dfrac{k}{4\pi} \int_{\mathcal{M}} tr\left(a \wedge da +\dfrac{2}{3} a\wedge a \wedge a\right)
\\ = \dfrac{k}{4\pi} \int_{\mathcal{M}} \epsilon^{\mu \nu \rho}
\left(a^{\underline{a}}_{\mu}\partial_{\nu} a^{\underline{a}}_{\rho}
+\dfrac{2}{3} f_{\underline{a}\underline{b}\underline{c}}a^{\underline{a}}_{\mu}a^{\underline{b}}_{\nu}a^{\underline{b}}_{\rho} \right)
\end{eqnarray}
In this expression, the gauge field now takes values in the Lie algebra
of the group G. $f_{\underline{a}\underline{b}\underline{c}}$ are the
structure constants of the Lie algebra which are simply $\epsilon_{\underline{a}\underline{b}\underline{c}}$ for the case of SU(2). 
For the case of SU(2), we thus have a gauge field $a^{\underline{a}}_{\mu}$, where the underlined indices run from 1 to 3. A matter field transforming in the spin-j representation of the SU(2) gauge group
a will couple to the combination $a^{\underline{a}}_{\mu} x_{\underline{a}}$ , where $x_{\underline{a}}$ are the three
generator matrices of su(2) in the spin-j representation. For
gauge group G and coupling constant k (called the ‘level’),
we will denote such a theory by $G_k$ . In this paper, we will be
primarily concerned with $SU(2)_{k}$ Chern-Simons theory.
To see that Chern-Simons theory is a TQFT, first note that
the Chern-Simons action (24) is invariant under all diffeomor-
phisms of M to itself, $f : \mathcal{M} \rightarrow \mathcal{M}$. The differential form
notation in (\ref{24}) makes this manifest, but it can be checked
in coordinate form for $x^{\mu}\rightarrow f ^{\mu} (x)$. Diffeomorphism invariance stems from the absence of the metric tensor in the
Chern-Simons action. Written out in component form, as in
(\ref{24}), indices are, instead, contracted with $\epsilon^{\mu \nu \lambda}$ .
Before analyzing the physics of this action (\ref{24}), we will
make two observations. First, as a result of the presence 
of $\epsilon^{\mu \nu \lambda}$, the action changes sign under parity or time-reversal transformations.
In this paper, we will concentrate, for the 
most part, on topological phases which are chiral, i.e. which 
break parity and time-reversal symmetries. These are the 
phases which can appear in the fractional quantum Hall ef- 
fect, where the large magnetic field breaks P , T . 
Secondly, the Chern-Simons action is not quite fully invariant under gauge transformations
$a_{\mu} \rightarrow ga_{\mu}g^{-1} + g\partial_{\mu}g^{-1}$,
where $g: \mathcal{M} \rightarrow G$ is any function on the manifold taking values in the group G. On 
a closed manifold, it is only invariant under 'small' gauge transformations. Suppose that the manifold
$\mathcal{M}$ is the 3-sphere, $S^{3}$. Then, gauge transformations are maps $S^3 \rightarrow G$, which can be classified
topologically according to its homotopy $\pi_{3}(G).$ For any simple compact group G, $\pi_{3}(G) = \mathbb{Z}$,
so gauge transformations can be classified according to their 'winding number'. Under a gauge transformation with winding m,
\begin{equation}\label{25}
 S_CS[a] \rightarrow S_CS [a] + 2\pi km
\end{equation}
While the action is invariant under 'small' gauge transformations, which are continuously connected to the identity and have m = 0,
it is not invariant under 'large' gauge transformations $(m \neq 0).$ However, it is sufficent for $\exp(iS)$ to be gague invariant,
which will be the case so long as we require that the level k be an integer. The requirement that the level k be an integer
is an example of the highly rigid structure of TQFTs. A small pertubation of the microscopic Hamiltonian cannot continuously 
change the value of k in the effective low energy theory; only a pertubation which is large enough to change k by
an integer can do this.
\paragraph{} The failure of gauge invariance under large gauge transformations is also reflected in the properties of
Chern-Simons theory on a surface with boundary, where the Chern-Simons action is gauge invariant only up to a
surface term. Consequently, there must be gapless degrees of freedom at the edge of the system whose dynamics is dictated
by the requirement of gauge invariance of the combined bulk and edge.
\paragraph{} To unravel the physics of Chern-Simons theory, it is useful to specialize to the case
in which the spacetime manifold $\mathcal{M}$ can be decomposed into a product of a spatial surface and time
$\mathcal{M} = \Sigma \times \mathbb{R}.$ On such a manifold, Chern-Simons theory is a theory of the ground states of a 
topologically-ordered system on $\Sigma$. There are no excited states in Chern-Simons theory because the Hamiltonian vanishes.
This is seen most simply in $a_0 = 0$ gauge, where the momentum canonically conjugate to $a_1$ is $-\dfrac{k}{4\pi}a_{2}$,
and the momentum canonically conjugate to $a_2$ is $\dfrac{k}{4\pi}a_{1}$ so that
\begin{equation}\label{(26)}
 \mathcal{H} = \dfrac{k}{4\pi} tr\left(a_2\partial_0 a_1 - a_1 \partial_0 a_2 \right) - \mathcal{L} = 0
\end{equation}
Note that this is a special feature of an ation with a Chern-Simons term alone. If the action had both a Chern-Simons and
a Yang-Mills term, then the Hamiltonian would not vanish, and the theory would have both ground states and excited states
with a finite gap. ince the Yang-Mills term is sublead-
ing compared to the Chern-Simons term (i.e. irrelevant in a
renormalization group (RG) sense), we can forget about it at
energies smaller than the gap and consider the Chern-Simons
term alone.
Therefore, when Chern-Simons theory is viewed as an ef-
fective field theory, it can only be valid at energies much
 smaller than the energy gap. As a result, it is unclear, at the
moment, whether Chern-Simons theory has anything to say
about the properties of quasiparticles – which are excitations
 above the gap – or, indeed, whether those properties are part
  of the universal low-energy physics of the system (i.e. are
 controlled by the infrared RG fixed point). Nevertheless, as
  we will see momentarily, it does and they are.
 Although the Hamiltonian vanishes, the theory is still not
trivial because one must solve the constraint which follows by
varying $a_{0}$ . For the sake of concreteness, we will specialize to
the case G =SU(2). Then the constraint reads:
\begin{equation}\label{27}
 \epsilon_{ij} \partial_{i}a^{\underline{a}}_{j} + f^{\underline{a} \underline{b}\underline{c}}a^{\underline{b}_{1} 
 a^{\underline{b}}_{2}} = 0
\end{equation}
where i,j = 1,2. The left-hand side of this equation is the field strength of the gauge field $a^{\underline{a}}_{i}$,
$\underline{a} = 1,2,3$ is an $\mathfrak{su}$(2) index. Since the field strength must vanish, we can 
always perform a gauge transformation so that $a^{\underline{a}}_{i} = 0$ locally. Therefore this theory has 
no local degrees of freedom. 
However, for some field configurations satisfying the constraint,
there may be a global topological obstruction which prevents
us from making the gauge field zero everywhere. Clearly, this
can only happen if $\Sigma$ is topologically non-trivial.
The simplest non-trivial manifold is the annulus, which is
topologically equivalent to the sphere with two punctures.
Following Elitzur et al., 1989\footnote{include this reference}, let us take coordinates
$(r, \phi)$ on the annulus, with $r_1 \lt r \lt r_{2}$ , and let t be time.
Then we can write $a_\mu = g\partial_{\mu}g^{-1}$ , where
\begin{equation}\label{28} 
 g(r,\phi,t)= e^{i\omega(r,\phi,t)}e^{i \dfrac{\phi}{k}\lambda{(t)}}
\end{equation}
where $\omega(r,\phi,t)$ and $\lambda{(t)}$ take values in the Lie algebra $\mathfrak{su}(2)$
and $\omega(r,\phi,t)$ is a single-valued function of $\phi$. The functions $\omega$ and $\phi$
are the dynamical variables of Chern-Simons theory on the annulus. Substituting (\ref{28})
into the Chern-Simons action, we see that it now takes the form:
\begin{equation}\label{29}
 S = \dfrac{1}{2\pi}\int dt tr(\lambda \partial_{t} \Omega)
\end{equation}
where $\Omega(r,t) = \int^{2\pi}_{0} d\phi (\omega(r_1,\phi,t) - \omega(r_2, \phi,t)).$ Therefore $\Omega$ 
is canonically conjugate to $\lambda$. By a gauge transformation, we can always rotate $\lambda$ and $\Omega$ so that they
are along the 3 direction in $\mathfrak{su}(2)$, i.e. $\lambda = \lambda_3 T^3$, $\Omega = \Omega_3 T^3$.
  Since it is defined through the exponential in $(\ref{28})$, $\Omega_3$ takes values in $[0,2\pi].$
 Therefore, its canonical conjugate $\lambda_3$ is quantized to be an integer from the definition of $\lambda$ in (\ref{28}),
 we see that $\lambda_3 \equiv  \lambda_3 + 2k.$ However, by a gauge transformation given by a rotation around the 1-axis, we 
 can transform $\lambda \rightarrow -\lambda$. Hence, the independent allowed values of $\lambda$ are 0,1,$\cdots$,k.
  \paragraph{} On the two-punctured sphere, if one puncture is of type a,
  the other puncture must be of type $\bar{a}$. (If the topological charge at one puncture is measured along a loop around
  the puncture - e.g. by a Wilson loop then the lopp can be deformed so that it goes around the other puncture, but in the
  opposite direction. Therefore, the two punctures, necessarily have conjugate topological charges.)
  For SU(2), $a = \bar{a}$, so both punctures have the same topological charge. 
  Therefore, the restriction to only k+1 different possible allowed boundary condition $\lambda$ for the two-punctured 
  sphere implies that there are k+1 different quasiparticle types in $SU(2)_{k}$ Chern-Simons theory. As we will describe in a later
  subsection, these allowed quasiparticle types can be identified with the $j = 0,\dfrac{1}{2},\cdots,\dfrac{k}{2}$
  representations of the $SU(2)_{2}$ Kac-Moody algebra. 