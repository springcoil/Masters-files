\documentclass[a4paper,10pt]{article}
\usepackage[utf8]{inputenc}
\usepackage{amssymb, amsmath, textcomp}

\newtheorem{theorem}{Theorem}[section]
\newtheorem{lemma}[theorem]{Lemma}
\newtheorem{proposition}[theorem]{Proposition}
\newtheorem{corollary}[theorem]{Corollary}
\newcommand{\lt}{<}
\newcommand{\gt}{>}

\newenvironment{proof}[1][Proof]{\begin{trivlist}
\item[\hskip \labelsep {\bfseries #1}]}{\end{trivlist}}
\newenvironment{definition}[1][Definition]{\begin{trivlist}
\item[\hskip \labelsep {\bfseries #1}]}{\end{trivlist}}
\newenvironment{example}[1][Example]{\begin{trivlist}
\item[\hskip \labelsep {\bfseries #1}]}{\end{trivlist}}
\newenvironment{remark}[1][Remark]{\begin{trivlist}
\item[\hskip \labelsep {\bfseries #1}]}{\end{trivlist}}

\newcommand{\qed}{\nobreak \ifvmode \relax \else
      \ifdim\lastskip<1.5em \hskip-\lastskip
      \hskip1.5em plus0em minus0.5em \fi \nobreak
      \vrule height0.75em width0.5em depth0.25em\fi}
      \newcommand{\christoffel}[3][i]{\ensuremath{\Gamma_{#1#2}^{#3}}}
   \newcommand{\chris}[3][i']{\ensuremath{\Gamma_{#1#2}^{#3}}}
     
%opening
\title{Definition of a Connection and Christoffel symbols}
\author{Peadar Coyle}

\begin{document}

\maketitle

\begin{abstract}
Some notes on Christoffel symbols and definitions of a Connection, to help internalise the notions.
\end{abstract}
\section{Connections}
Firstly, let us solidify and recall some notation
The set of all vector fields $\Gamma(TM)$ takes on the structure of a module over the 
commutative algebra of smooth functions on M, denoted $C^{\infty}(M)$.
\begin{definition}
 An affine connection on M is the operation $\nabla$ which assigns to every vector field $\mathbf{X}$ a linear
 map but not necessarily a $C^{\infty}(M)$- linear map\footnote{There is a map which is linear over the reals but not necessarily
 over the ring of functions}
 $\nabla_{\mathbf{X}}$ on the space $\mathcal{O}$(M) of vector fields - we can say that the is a module of vector fields
 on the sheaf of rings. 
\begin{eqnarray}
 \nabla_{\mathbf{X}}(\lambda \mathbf{Y} + \mu \mathbf{Z}) \\
 = \lambda \nabla_{\mathbf{X}} \mathbf{Y} + \mu \nabla_{\mathbf{X}} \mathbf{Z}
\end{eqnarray}
for every $\lambda, \mu\;\in\;\mathbb{R}$
such that the following conditions are satisfied 
\begin{itemize}
 \item for arbitrary (smooth) functions f,g on M 
 \begin{equation}\label{2.4}
  \nabla_{f\mathbf{X} + g\mathbf{Y}}(\mathbf{Z}) = f \nabla_{\mathbf{X}}(\mathbf{Z}) 
  +g \nabla_{\mathbf{Y}}(\mathbf{Z})
 \end{equation}
this is called $C^{\infty}$-linearity
\item for arbitrary function f
\begin{equation}\label{2.5}
 \nabla_{\mathbf{X}}(f\mathbf{Y}) = (\nabla_{\mathbf{X}}f)\mathbf{Y} + f(\nabla_{\mathbf{X}}(\mathbf{Y})
\end{equation}
recall that $\nabla_{\mathbf{X}}f$ is just the usual directional derivative of a function f along a vector field
so we'll say $\nabla_{\mathbf{X}} = df$ 
\end{itemize}

\end{definition}
Let us write down explicit formulae in a given local co-ordinates $\left\lbrace x_i \right\rbrace$(i=1,$\cdots$,n)
on a manifold M. 
Let 
$\mathbf{X} = X^i\mathbf{e}_{i}= X^{i}\dfrac{\partial}{\partial x^{i}}$
$\mathbf{Y} = Y^i\mathbf{e}_{i} = Y^i \dfrac{\partial}{\partial x^i}$
So lets write 
\begin{equation}\label{2.6}
 \nabla_{\mathbf{X}}\mathbf{Y} = \nabla_{X^i\partial_i}Y^k\partial_k = X^{i}\left(\nabla_i\left(Y^k\partial_k\right)\right)
\end{equation}
where $\partial_i$ is merely the base vector field, and 
$\nabla_i = \nabla_{\partial_{i}}$, the $X^{i}$ comes up after the second equal sign in (\ref{2.6}) because it is merely
a function. According to (\ref{2.4})
\begin{equation}
 \nabla_i(Y^k \partial_k) = \nabla_i(Y^k)\partial_k + Y^k(\nabla_i \partial_k)
\end{equation}
Let us now decompose the vector field $\nabla_i \partial_k$ over the basis $\partial_i$:
\begin{equation}\label{2.7}
 \nabla_i \partial_k = \christoffel{k}{m} \partial_m
\end{equation}
(take the two subscripts on the l.h.s and put them at the bottom on the Christoffel symbol ($\Gamma$) and include a 
dummy index)
and 
\begin{equation}\label{2.8}
 \nabla_i(Y^k \partial_k) = \dfrac{\partial Y^{k}(x)}{\partial x^{i}}\partial_{k} + Y^{k}\christoffel{m}{k} \partial_m
\end{equation}
And to return to where we began
\begin{equation}\label{2.9}
 \nabla_{\mathbf{X}}(\mathbf{Y}) = X^{i}\dfrac{\partial Y^{m}}{\partial x^{i}}\partial_{m} + 
 X^{i}Y^{k}\christoffel{m}{k} \partial_m
\end{equation}
The coefficients $\left\lbrace \christoffel{m}{k}\right\rbrace$ are called \textit{Christoffel symbols} in coordinates
$\left\lbrace x_i \right\rbrace$. We can say that these coefficients define a covariant derivative or a \textbf{connection} 
  \subsection{Transformation of Christoffel symbols for an arbitrary connection}
 Let $\nabla$ be a connection on a manifold M. Let {\christoffel{k}{m}} be Christoffel symbols of this connection
 given in local coordinates $\left\lbrace x_i\right\rbrace$. Then according to (\ref{2.7}) and (\ref{2.8}) we have
 \begin{equation}
  \nabla_{\mathbf{X}}\mathbf{Y} = X^{m}\dfrac{\partial Y^{i}}{\partial x^{m}} \dfrac{\partial}{\partial x^{i}} +
  X^{m} \christoffel{m}{k} Y^{k} \dfrac{\partial}{\partial x^{i}}
 \end{equation}
and in particular 
$\christoffel{m}{k} \partial_i = \nabla_{\partial_{m}} \partial_k$
Let us use this relation to calculate Christoffel symbols in a new coordinates 
$\left\lbrace x^{i'} \right\rbrace$ 
\begin{equation*}
\Gamma_{m'k'}^{i'}\partial_{i'} = \nabla_{\partial_{m'}}\partial_{k'}
\end{equation*}
We have that $\partial_{m'} = \dfrac{\partial}{\partial x^{m'}}=
\dfrac{\partial x^{m}}{\partial x^{m'}}\dfrac{\partial}{\partial x^{m}} = \dfrac{\partial x^{m}}{\partial x^{m'}}\partial_m$
Hence due to properties (\ref{2.4}) and (\ref{2.5}) we have 
\begin{eqnarray*}
\Gamma_{m'k'}^{i'}\partial_{i'} = \nabla_{\partial_{m'}}\partial_{k'} = \nabla_{\partial_{m'}}\left(\dfrac{\partial x^{k}}
{\partial x^{k'}}\partial_{k}\right) = \left(\dfrac{\partial x^{k}}{\partial x^{k'}}\right)\nabla_{\partial_{m'}}\partial_k
+\dfrac{\partial}{\partial x^{m'}}\left(\dfrac{\partial x^{k}}{\partial x^{k'}}\right)\partial_k 
\\
= \left(\dfrac{\partial x^{k}}{\partial x^{k'}} \right)_{\nabla_{\dfrac{\partial x^{m}}{\partial x^{m'}}}\partial_{m}}\partial_k
+ \dfrac{\partial^{2} x^{k}}{\partial x^{m'}\partial x^{k'}}\partial_k
= \dfrac{\partial x^{k}}{\partial x^{k'}} \dfrac{\partial x^{m}}{\partial x^{m'}} \Gamma^{i}_{mk}\partial_{i}
+ \dfrac{\partial^{2}x^{k}}{\partial x^{m'} \partial x^{k'}}\partial_{k}
\\ = \dfrac{\partial x^{k}}{\partial x^{k'}}\dfrac{\partial x^{m}}{\partial x^{m'}} \Gamma^{i}_{mk} 
\dfrac{\partial x^{i'}}{\partial x^{i}}\partial_{i'} + 
\dfrac{\partial^{2} x^{k}}{\partial x^{m'} \partial x^{k'}} 
\dfrac{\partial x^{i'}}{\partial x^{k}}\partial_{i'}
\end{eqnarray*}
We get the transformation law: 
If $\left\lbrace \Gamma_{km}^{i}\right\rbrace$
 are Christoffel symbols fo the connection $\nabla$ in local coordinates $\left\lbrace x^i\right\rbrace$
and $\left\lbrace \Gamma_{k'm'}^{i'}\right\rbrace$ are Christoffel syombols of this connection in new local coordinates
$\left\lbrace x^{i'}\right\rbrace$ then
\begin{equation}\label{2.11}
 \Gamma_{k'm'}^{i'} = \dfrac{\partial x^{k}}{\partial x^{k'}}\dfrac{\partial x^{m}}{\partial x^{m'}}
 \dfrac{\partial x^{i'}}{\partial x^{i}}\Gamma_{km}^{i}
 + \dfrac{\partial^{2} x^{r}}{\partial x^{k'}\partial x^{m'}} \dfrac{\partial x^{i'}}{\partial x^{r}}
\end{equation}
\begin{remark}
 Christoffel symbols do not transform as a tensor. If the second term is euqal to zero, i.e. transformation 
 of coordinates are linear then the transformation rule above is the same as a transformation rule for tensors
 of the type  $\binom{1}{2}$. In general this ins note true. Christoffel symbols do not transform as a tensor under
 arbitrary non-linear coordinate transformations.
\end{remark}

  \section{What is a Geodesic}
 Let us recall the definition of a geodesic. 
 A \textbf{geodesic} on a smooth manifold M with an \textbf{affine connection} $\nabla$ is defined as a curve 
 $\gamma(t)$ such that \textbf{parallel transport} along the curve preserves the tangent vector to the curve, so 
 \begin{equation}\label{1}
  \nabla_{\dot{\gamma}} \dot{\gamma} = 0
 \end{equation}
at each point along the curve, where $\dot{\gamma}$ is the derivative w.r.t. t. More precisely, in order to define the 
covariant derivative of $\dot{\gamma}$ to a continuously differentiable vector field on an open set. 
However, the resulting value of \ref{1} is independent of the choice of extension.
    We can however write this in local form, 
    \begin{equation*}
     \dfrac{d^{2} x^{\lambda}}{dt^2} + \Gamma_{\mu \nu}^{\lambda} \dfrac{dx^{\mu}}{dt} \dfrac{dx^{\nu}}{dt} = 0,
    \end{equation*}
where $x^{\mu}(t)$ are the coordinates of the curve $\gamma(t)$ and $\Gamma_{\mu \nu}^{\lambda}$ are the Christoffel
symbols of the connection $\nabla$. This is just an ODE for the coordinates. It has a unique solution, given an
initial position and an initial velocity. Therefore, fromt he point of view of classical mechainics, geodesics can 
be thought of as trajectories of \textbf{free} \textbf{particles} in a manifold. Indeed, the equation 
$\nabla_{\dot{\gamma}}\dot{\gamma} = 0$ means that the acceleration of the curve has no components in the direction 
of the surface (the direction of the surface is the same as the direction of the affine connection). We can also 
say that the acceleration of the curve is perpendicular to the tangent plane of the surface at each point of the
curve. So the motion is completely determined by the bending of the surface. This also the idea of General Relativity where
particles move on geodesics and the bending is caused by gravity. 
\section{Bundles and Connections}
Let us review some of the fundamental and classical notions of Principal Bundles and Connections.
Let G be a Lie group. A principal G-bundle P over a smooth manifold X is a manifold with a (smooth) right G
action and orbit space $P/G = X$
   We demand that the action admits local product structures, i.e. it is locally
   equivalent to the product space U x G, where U is an open set in X. Then we have a fibration 
   $\pi: P \rightarrow X$. We say that P has structure group G.
 Three useful ways of defining a connection on such a bundle are:
 \begin{enumerate}
  \item As a field of 'horizontal subspaces' $H \subset TP$ transverse to the fibers of $\pi$
  That is, for each $p \in P$ we have the decomposition
  \begin{equation*}
   TP_{p} = H_p \oplus T(\pi^{-1}(x))
  \end{equation*}
where $\pi(p) = x$. The field of subspaces is required to be preserved by the action of G on P. 
\item As a 1-form A on P with values in the Lie algebra $\mathfrak{g}$ of G i.e. a section of the 
vector bundle 
$T^*P \otimes \mathfrak{g}$
over P. Again we require this to be invariant under G, acting by a combination of the given action on P and
the adjoing action on $\mathfrak{g}$. Also, A should be restricted to the canonical right-invariant form on the 
fibres.
\item For any linear representation of G, on $\mathbb{C}^n$ or $\mathbb{R}^n$, we get a vector bundle E over X 
associated with P and the representation. The fibres of E are copies of $\mathbb{C}^n$ or $\mathbb{R}^n$ respectively. 
Conversely, if given a vector bundle E we can make a principal bundle. For example, if E is a complex
n-plane bundle we get a principal bundle P with the structure group $GL(n,\mathbb{C})$ by taking the set of 
all 'frames' in E. A point in the fibre of P over x $\in$ X is a set of basis vectors for $E_x$. Additional algebraic
structure on E yields a principal bundle with a smaller structure group. For example is E is a complex vector bundle with
a Hermitian metric we get a principal bundle U(n) bundle of orthonormal frams.e 
   \paragraph{} For classical groups, (automorphisms of a vector space preserving some linear algebraic structure)
   the concepts of principal and vector bundles are completely equivalent. 

 \end{enumerate}
Now given a vector bundle E (real or complex) a connection on the frame bundle can be defined by a \textbf{covariant derivative}
on E, that is a linear map:
\begin{equation}
 \nabla: \Omega_{X}^{0}(E) \rightarrow \Omega_X^1(E)
\end{equation}
Here we introduce the notation $\Omega^p_X(E)$ to denote sections of 
$\bigwedge^pT^*X \otimes E$ - the p-forms with values in E. The map 
$\nabla$ is required to satisfy the Leibniz rule 
$\nabla(f.s) = df.s + f\nabla s$ for any f a real or complex function on X and s $\in \Gamma(E)$, where $\Gamma(E)$ 
denotes sections on E. 
   To understand this definition, consider a tangent vector v to X at a point x. If s is a section of E we can make 
   the contraction $\langle \nabla s, v \rangle \in E_x$ which is to be thought of as the derivative of s in the 
   direction v at x. If E has additional algebraic structure we can require the covariant derivative to be compatible 
   with this; for example if E has a metric, a compatible connection is one for which $\nabla(s,t) = (\nabla s, t) + (s, 
   \nabla t)$, for all sections s,t. 
      One see the equivalence of these viewpoints as follows to go from (2) to (1) we regard $H_p$ (the horizontal 
    space) to be the kernel of $A_p: TP_{p} \rightarrow \mathfrak{g}$.
  To go from (3) to (1) we first observe that $\nabla$ is a local operator; that is, if two sections $s_1$, $s_2$
  agree on an open set U in X, so also do $\nabla s_1$ $\nabla s_2$. 
  This follows from Leibnitz rule by considering $\phi s_i$, where $\phi$ is a cutoff function. 
 Then we say that a local section $\delta$ of the frame bundle (i.e. a collection of sections $s_1,\cdots, s_n$ of E)
 is horizontal at x in X if all the $\nabla s_i$ vanish at x. Finally we define $H_p$ to be the tangent space to a 
 horizontal section $\delta$ through p, regarded as a submanifold of P. 
 \end{document}

